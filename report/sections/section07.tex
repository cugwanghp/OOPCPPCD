\section{总结}
    本次实习,使我收获颇丰。但我也深刻知道,数字摄影测量是一门严密且高深的学问,本次实习所涉及到的内容,不过沧海之一粟。循序渐进的实习、课设课程使我逐步理解从解析摄影测量到
    数字摄影测量的过程。实践与理论的结合使我不得不关注到算法的每一处细节,因为在实践过程中,差之毫厘,失之千里。很多次当我发现结果并不理想时,通过仔细检查代码,向同学请教,才发现是我对于原理理解有误。而这极大的
    促进了我对于摄影测量算法的进一步理解。

    本次实习我学会了ERDAS中LPS模块的使用,了解了空中三角测量的处理过程,掌握了立体像对构建DTM的具体流程。本次实习通过\textit{Python}编写了\textit{Moravec}算子、\textit{Harris}算子和\textit{Hough}变换,
    探究了不同颜色波段、算法参数、影像增强、影像变换、影像噪声对于点特征提取结果的影响,也探究了算法参数、差分算子对于线特征提取结果的影响。本次实习亦编程实现相关系数法同名点匹配,分析并理解算法参数对于同名点匹配的影响。
    
    最后,衷心感谢宋老师、乐老师、徐老师为我们精心准备实习内容,使我们能够非常方便的理解和实践摄影测量相关算法。同时也非常感谢各位老师在空调罢工的日子里顶着炎炎夏日陪伴我们,为我们答疑解惑,
    使我们对于摄影测量算法有着进一步的理解。