
\section{特征提取:Hough变换}
\subsection{简介}

    \textit{Hough}变换是一种图像处理中的特征提取算法。其实质即将坐标空间中的点变换到参数空间中,通过统计参数空间中点的数目反推坐标空间中对应几何特征物。
    常用的\textit{Hough}变换的特征检测有线特征检测、圆特征检测等。

\subsection{原理}
    \subsubsection{符号约定}
        
    \begin{table}[H]
        \centering
        \begin{tabular}{cl}
            \toprule
              符号  & 说明 \\
            \midrule
                $\rho$  & \textit{Hough}参数空间中的纵坐标,在\textit{XOY}坐标系中代表某点/线到原点的距离  \\
                $\rho_0$  & 某一条直线在\textit{Hough}参数空间中的纵坐标  \\
                $\alpha$  & \textit{Hough}参数空间中的横坐标,在\textit{XOY}坐标系中代表直线与纵轴夹角  \\
                $\alpha_0$  & 某一条直线在\textit{Hough}参数空间中的横坐标  \\
                $k$  & 在\textit{XOY}坐标系中直线的斜率  \\
                $b$ & 在\textit{XOY}坐标系中直线的纵截距\\
                $x_0$ & 某一点在\textit{XOY}坐标系中的横坐标 \\
                $y_0$ & 某一点在\textit{XOY}坐标系中的纵坐标 \\
            \bottomrule
        \end{tabular}
        \caption{符号约定}
        \label{hough_symbol}
    \end{table}

    \subsubsection{基础概念}
        \begin{theorem}{直线的斜截式方程}
            在式中,\textit{k}表示直线斜率,其含义为直线与横坐标轴夹角的正切值;\textit{b}表示截距,其为直线与纵坐标轴的截距。
            \begin{equation}
                \label{line_kb}
                y=kx+b
            \end{equation}
            而直线的斜截式方程无法表示垂直于横坐标轴的直线,同时接近于垂直横坐标轴的直线,其斜率极大,不易处理。
        \end{theorem}

        \begin{theorem}{直线的法线式方程}
            在式中,$\rho$表示直线到原点的距离,$\alpha$表示直线与纵坐标的夹角。
            \begin{equation}
                \label{line_rho}
                \rho = x cos\alpha+y sin\alpha
            \end{equation}
            该直线方程可以表示平面内任意一条直线。
        \end{theorem}

        \paragraph{转换}斜截式方程可以很轻易的转换为法线式方程。其中:
        \begin{equation}
            \label{kx2ra}
            \left\{
                \begin{aligned}
                    \rho &= \frac{|b|}{\sqrt{k^2+1}} \\
                    \alpha&= \frac{\pi}{2}-arctan(k)
                \end{aligned}
            \right.
        \end{equation}

    \subsubsection{算法原理}
        \begin{enumerate}
            \item Hough变换:
            
            \hspace{20pt}Hough变换即将原\textit{XOY}平面中的点变换到参数空间中。\textbf{参数空间}指直线参数之空间,其横纵坐标轴为式(\ref{line_rho})中的参数$\alpha, \rho$。之所以使用直线的法线式方程(式(\ref{line_rho}))而不使用
                            直线的斜截式方程(式(\ref{line_kb})),在于直线的法线式方程可以表示平面直角坐标系中所有的直线,而斜截式方程无法表示垂直于横坐标轴的直线。
                \begin{itemize}
                    \item \textbf{线变换:}该变换将原\textit{XOY}平面中的直线变换为\textit{Hough}参数空间中的一个点。

                        \begin{equation}
                            XOY: \ \rho_0 = x cos\theta_0+y sin\theta_0 \ \rightarrow \ Hough: \ (\theta_0,\rho_0)
                        \end{equation}
                    \item \textbf{点变换:}该变换将原\textit{XOY}平面中的点变换为\textit{Hough}参数空间中的一条正弦曲线。
                    
                    \begin{equation}
                        XOY: \ (x_0,y_0) \ \rightarrow \ Hough: \ \rho = x_0 cos\theta+y_0 sin\theta
                    \end{equation}
                    
                \end{itemize}

            \hspace{20pt}由此,将原平面中所有的点根据方程(\ref{line_rho})变换至\textit{Hough}参数空间中。

            \item 统计投票:
            
            \hspace{20pt} \textit{Hough}参数空间每一个点代表一条直线,而每一条正弦曲线则代表一个点。若\textit{Hough}参数空间中多条正弦曲线交于一点$P_{Hough}$,则该若干条曲线在\textit{XOY}平面中所代表的点均在$P_{Hough}$所代表的直线上。
            因此仅需找到\textit{Hough}参数空间中多条正弦曲线的交点即可。统计相交次数,根据阈值筛选出符合条件的点。
                        
            \item 反变换:
            
            \hspace{20pt}将\textit{Hough}参数空间中上一步得到的点根据方程(\ref{line_rho})反变换到\textit{XOY}平面中。

        \end{enumerate}

    
\subsection{程序编写}
    \subsubsection{开发环境}
        程序基于\textit{Python}语言编写,使用\textit{Pycharm}进行编写。程序中使用\textit{Opencv}库对影像进行读写,使用\textit{Numpy}库进行矩阵运算,使用\textit{Matplotlib}库进行结果可视化。
    \subsubsection{算法流程}
        \begin{enumerate}
            \item [(1)]影像预处理:这一步包括将彩色影像转换为灰度影像,影像几何变换,影像加噪声等处理。
            \item [(2)]边缘检测:对影像进行边缘检测以凸显线特征。
            \item [(3)]二值化:\textit{Hough}变换要求输入影像为二值影像,对边缘检测算子计算后影像二值化。
            \item [(4)]\textit{Hough}变换:对二值化后影像中的每一个非背景点,将其根据式(\ref{line_rho})变换至\textit{Hough}空间。
                        程序进行\textit{Hough}时涉及离散采样操作。由\textit{XOY}空间通过式(\ref{line_rho})变换至\textit{Hough}空间时,原本的整型会变为浮点型。
                        若欲统计相交点数,则需进行离散化的格网划分。离散化的对象为$\theta, \rho$。据此设计的参数为:
                        \begin{itemize}
                            \item 角度离散化参数(angle\_part):该参数将$0-\pi$划分为angle\_part份,每一份的$\theta$将被认为是相等的。
                            \item 距离离散化参数(rho\_res):该参数将$0-\rho_{max}$划分为rho\_res份,每一份$\rho$将被认为是相等的。
                            \item 投票阈值(thres):该参数为步骤(5)之阈值。
                        \end{itemize}

            \item [(5)]投票统计:统计\textit{Hough}空间中每一个交点相交直线的数目,并根据阈值提取符合条件的交点。
            \item [(6)]反\textit{Hough}变换:将(5)中得到的交点根据式(\ref{line_rho})反变换至\textit{XOY}平面。
        \end{enumerate}

        \begin{figure}[H]
            \centering 
            \includegraphics[width=\textwidth]{hough/stream.png}
            \caption{流程图}
            \label{hough_stream}
        \end{figure}

    \subsubsection{核心代码说明}
        \begin{enumerate}
            \item \textit{Hough}变换代码
            
            \begin{figure}[H]
                \centering 
                \includegraphics[width=0.8\textwidth]{hough/trans1.png}
                \caption{变换代码}
                \label{hough_transcode}
            \end{figure}

            其中$x,y$为二值化图像中每一个非背景点的坐标,$theta$为将$0-\rho_{max}$划分为rho\_res份产生的向量,不妨设
            $\bm{X}=[x_1,x_2, \cdots, x_n], \bm{Y}=[y_1,y_2, \cdots , y_n],\bm{\theta} = [\theta_1,\theta_2, \cdots ,\theta_k]$,则上述代码表示
            \begin{equation}
                \label{calrho}
                    \begin{aligned}
                        \bm{\rho}&=\bm{X}^T \times cos\bm{\theta} + \bm{Y}^T \times sin\bm{\theta} \\
                                &=\begin{bmatrix} x_1 cos\theta_1+y_1 sin\theta_1 & x_1 cos\theta_2+y_1 sin\theta_2 & \cdots & x_1 cos\theta_k+y_1 sin\theta_k \\  x_2 cos\theta_1+y_2 sin\theta_1 & x_2 cos\theta_2+y_2 sin\theta_2 & \cdots & x_2 cos\theta_k+y_2 sin\theta_k \\ \vdots & \vdots & \ddots &\vdots \\  x_n cos\theta_1+y_n sin\theta_1 & x_n cos\theta_2+y_n sin\theta_2 & \cdots & x_n cos\theta_k+y_n sin\theta_k  \end{bmatrix}\\
                                &=\begin{bmatrix} \rho_1(\theta_1) & \rho_1(\theta_2) & \cdots & \rho_1(\theta_k) \\ \rho_2(\theta_1) & \rho_2(\theta_2) & \cdots & \rho_2(\theta_k) \\ \vdots & \vdots & \ddots &\vdots \\ \rho_n(\theta_1) & \rho_n(\theta_2) & \cdots & \rho_n(\theta_k) \\ \end{bmatrix}
                    \end{aligned}
            \end{equation}
            其中:$\rho_i(\theta_j)$代表第i个点第j个角度的$\rho$,由此则将\textit{XOY}平面坐标转换为\textit{Hough}参数空间坐标。

            \item 生成\textit{Hough}空间矩阵
            
            \begin{figure}[H]
                \centering 
                \includegraphics[width=0.8\textwidth]{hough/trans2mat.png}
                \caption{生成矩阵代码}
                \label{hough_mat}
            \end{figure}

            针对上一步(式(\ref{calrho}))得到的坐标值,将其映射到一个矩阵中计算交点位置。
            该步骤首先创建一个空的0值矩阵,其长为angle\_part,宽为angle\_part。
            遍历上一步得到的点坐标,找到其在矩阵中的位置,并将该位置处的值加1,直到所有点得到遍历。
            若某一点为交点,则其在矩阵中的值大于1。根据阈值筛选处符合的交点,反变换至\textit{XOY}平面空间。


        \end{enumerate}

    \subsubsection{函数调用}
        本次实习中,程序将\textit{Hough}变换封装为一个类。因此其调用首先需要申请一个类对象,
        随后调用对象的\textit{HoughTrans()}函数,得到图像中符合条件的线之端点坐标。

        在申请类对象时,需要输入两个参数:
        \begin{itemize}
            \item 影像:二值化后的影像,数据为\textit{Numpy}
            \item \textit{Hough}变换参数:为一个字典,其定义如下:
            \begin{itemize}
                \item 角度离散化参数(angle\_part):该参数将$0-\pi$划分为angle\_part份,每一份的$\theta$将被认为是相等的。
                \item 距离离散化参数(rho\_res):该参数将$0-\rho_{max}$划分为rho\_res份,每一份$\rho$将被认为是相等的。
                \item 投票阈值(thres):该参数为\textit{Hough}空间投票矩阵的阈值,只有大于阈值的交点才会输出其坐标。
            \end{itemize}
        \end{itemize}

        最后使用\textit{Matplotlib}库绘制影像和检测到特直线,可视化结果。

\subsection{结果分析}
    \subsubsection{结果展示}
    
    \begin{figure}[H]
        \centering
        \subfigure[提取结果]
        {
            \includegraphics[width=0.6\textwidth]{hough/result/result.png}
        }
        \subfigure[Hough空间影像]
        {
            \label{houghmatpic}
            \includegraphics[width=0.3\textwidth]{hough/result/houghmat.png}
        }
        \caption{提取结果}
        \label{hough_result}
    \end{figure}

    由于其他实验影像线性特征并不明显,因此本实验仅使用其中一幅影像作为实验影像。

    从实验结果分析,影像部分线特征基本被提取出,部分未提取出的直线可能是由于阈值设置的问题。图(\ref{houghmatpic})为原
    影像变换到\textit{Hough}空间后的影像,纵轴为$\rho$,横轴为$\theta$,其像元值为该点相交的次数。

    \subsubsection{参数对于提取结果的影像}

    \begin{enumerate}
        \item 角度$\theta$划分
        
        \begin{figure}[H]
            \centering
            \subfigure[划分为45份]
            {
                \includegraphics[width=0.22\textwidth]{hough/result/theta45.png}
            }
            \subfigure[划分为90份]
            {
                \includegraphics[width=0.22\textwidth]{hough/result/theta90.png}
            }
            \subfigure[划分为180份]
            {
                \includegraphics[width=0.22\textwidth]{hough/result/theta180.png}
            }
            \subfigure[划分为360份]
            {
                \includegraphics[width=0.22\textwidth]{hough/result/theta360.png}
            }
            \caption{提取结果}
            \label{hough_thetapic}
        \end{figure}

        \hspace{20pt}角度$\theta$划分是指将角度$\pi$划分为\textit{k}份。理论上划分的越精细,探测到的直线越多。本节分别将角度$\pi$
        划分为$45,90,180,360$份,分析结果。

        \hspace{20pt}从实验结果上看,随着角度划分的份数越小,算法检测到的直线也越多。而当划分份数达到一定时,更细的划分也无法使新的直线探测被检测出。
        此处推荐划分为$180$份即可。

        \item 投票阈值
        
        \begin{figure}[H]
            \centering
            \subfigure[投票阈值70]
            {
                \label{hough_threspic70}
                \includegraphics[width=0.22\textwidth]{hough/result/thres70.png}
            }
            \subfigure[投票阈值80]
            {
                \label{hough_threspic80}
                \includegraphics[width=0.22\textwidth]{hough/result/thres80.png}
            }
            \subfigure[投票阈值90]
            {
                \includegraphics[width=0.22\textwidth]{hough/result/thres90.png}
            }
            \subfigure[投票阈值100]
            {
                \includegraphics[width=0.22\textwidth]{hough/result/thres100.png}
            }
            \caption{不同投票阈值提取结果}
            \label{hough_threspic}
        \end{figure}

        \hspace{20pt}投票阈值是指在\textit{Hough}空间矩阵中,相交超过一定次数的值。其可以理解为在该条直线上经过了多少个像素点。本节
        分别设置投票阈值为$70,80,90,100$,分析结果。

        \hspace{20pt}从结果分析,随着投票阈值的增加,提取到的直线逐渐增多。但投票阈值过小则会导致一些不明显的直线被提取为直线。
        如图(\ref{hough_threspic70},\ref{hough_threspic80})中的红色直线,其并不具备直线的特征,但其经过的点数达到了一定值亦被提取为直线。

    \end{enumerate}

    \subsubsection{不同差分算子对提取结果影响}
    \begin{figure}[H]
        \centering
        \subfigure[Sobel算子]
        {
            \includegraphics[width=0.3\textwidth]{hough/result/sobel.png}
        }
        \subfigure[Laplacian算子]
        {
            \includegraphics[width=0.3\textwidth]{hough/result/lap.png}
        }
        \subfigure[Canny算子]
        {
            \includegraphics[width=0.3\textwidth]{hough/result/canny.png}
        }
        \caption{差分算子提取结果}
        \label{hough_delta}
    \end{figure}
    \textit{Hough}变换仅接受二值图像。因此需要先通过差分算子提取边界以二值化。本节分别使用\textit{Sobel,Laplacian,Canny}三种
    算子对原灰度影像二值化。

    实验结果表明,效果最好而差分算子为\textit{Canny}算子,其边界完整,无较多其他杂物影响。其次为\textit{Laplacian}算子,其边界完整,
    但其有较多纹理影响。当投票阈值较小时,这些纹理可能造成较多的线提取错误。效果最差为\textit{Sobel}算子,其错误线提取较多,
    有较多噪声影响。因此本实验推荐使用\textit{Canny}算子作为线特征提取前的差分算子。

\subsection{小结}
    本次实习通过编写程序实现\textit{Hough}变换,并针对其参数和差分算子进行了实验,深刻了解了\textit{Hough}变换的原理和适用条件,
    基本达到实习目的。