
\section{实习一C++语法基础}
\subsection{闰年判断}
    \subsubsection{题目}
        输入年份,判断是否为闰年,输出判断结果。
    \subsubsection{流程图}
        \begin{figure}[H]
            \centering
            \subfigure[11]
            {
                \includegraphics[width=0.4\textwidth]{cuglogo.png}
            }
            \subfigure[22]
            {
                \includegraphics[width=0.4\textwidth]{cuglogo.png}
            }
            \caption{流程图}
        \end{figure}

        简述程序流程。
        \begin{enumerate}
            \item 1
            \item 2
            \item 3
        \end{enumerate}

    \subsubsection{运行结果}
        \begin{figure}[H]
            \centering
            \includegraphics[width=0.4\textwidth]{cuglogo.png}
        \end{figure}

    \begin{table}[H]
        \centering
        \begin{tabular}{ccccccc}
            \toprule
            影像    & Xo         & Yo         & Zo       & Omega  & Phi    & Kappa   \\
            \midrule
            col90p1 & 666700.000 & 115900.000 & 8800.000 & 0.0000 & 0.0000 & 90.0000 \\
            col91p1 & 666700.000 & 119400.000 & 8800.000 & 0.0000 & 0.0000 & 90.0000 \\
            col92p1 & 666800.000 & 122900.000 & 8800.000 & 0.0000 & 0.0000 & 90.0000 \\
            \bottomrule
        \end{tabular}
        \caption{影像外方位元素}
        \label{lps_expos}
    \end{table}

\subsection{第2题}
    \subsubsection{题目}
    \subsubsection{流程图}
    \subsubsection{运行结果}

\subsection{小结}
本次实习遇到的问题,解决方案等,基本达到实习目的。
